\usepackage{amsmath, amssymb, amsthm, amsfonts, algpseudocode, algorithm, bbm, color, fixmath, float, graphicx, hyperref, listings, mathrsfs, mathtools, subfig, times} 

%Needed commands
\newcommand*{\w}{\mathbf{w}}
\newcommand*{\x}{\mathbf{x}}
\newcommand*{\y}{\mathbf{y}}
\newcommand*{\z}{\mathbf{z}}
\newcommand*{\R}{\mathbb{R}}
\newcommand*{\E}{\mathbb{E}}
\newcommand*{\0}{\mathbf{0}}
\newcommand*{\minimizer}{\mathbf{x}^*}
\newcommand*{\dprime}{{\prime\prime}}
\newcommand{\li}[1]{\lstinline[prebreak=]!#1!}
\newcommand{\pseudoli}[1]{\lstinline[style=pseudo]!#1!}
\newcommand{\trp}{^{\mathsf T}} 
\newcommand{\im}{{i\mkern1mu}}
\newcommand{\Real}{\mathchardef\Re="023C}
\newcommand{\Imag}{\mathchardef\Im="023D}
\newcommand\norm[1]{\left\lVert#1\right\rVert}
\newcommand*\diff{\mathop{}\!\mathrm{d}}
\newcommand*\Eval[3]{\left.#1\right\rvert_{#2}^{#3}}

%Operators
\DeclareMathOperator{\argmin}{argmin}
\DeclareMathOperator{\argmax}{argmax}

%Link set up
\hypersetup{
    colorlinks=true, %set true if you want colored links
    linktoc=all,     %set to all if you want both sections and subsections linked
    linkcolor=blue,  %choose some color if you want links to stand out
    pdftitle={RL Notes},
    pdfpagemode=FullScreen
}

\endinput