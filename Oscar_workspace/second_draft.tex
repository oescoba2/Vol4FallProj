\documentclass[11pt]{amsart}
%%% WARNING: Do NOT change the page size, fonts, or margins!  Penalties will apply.


\usepackage{graphicx}
\usepackage{amssymb, amsmath, amsthm}
\usepackage{places}
\usepackage{wrapfig,floatrow}
\usepackage{varwidth, caption, subcaption} %enables \FloatBarrier, which prevents figures and tables from going below it.
%\usepackage{hyperref} %makes cross references into hyperlinks. 
\input{commands}

%%% WARNING: Do NOT change the page size, fonts, or margins!  Penalties will apply.
%%% WARNING: Do NOT change the page size, fonts, or margins!  Penalties will apply.
% The Effect of Chemotherapy and Immunological Response on Breast Cancer Growth
\title{Breast Cancer Model with Chemotherapy and Immunological Response}
\author{R. Gee, H. Fetzer, N. Suyama, J. Humpherys, O.J. Escobar}
\date{October 29 2024} % or use \today

\begin{document}
\maketitle % this actually makes the title

\begin{abstract}
Place abstract here. The abstract summarizes in one paragraph the main question and conclusions draw from your investigation.
\end{abstract}

%% First Section
\section{Background/Motivation}

Mathematical oncology employs math to study cancer.
``It [serves] as a bridge between $\ldots$ the biologist, and the practicing clinician" (Dr. Rockne and MD Scott) \cite{IntroMathOnc}.
Tumor growth modeling is an application of mathematical oncology, which seeks to understand and model the properties that govern cancer growth, and the relationship between cancer, treatment and the immune system.

The purpose of tumor growth modeling is to: (1) develop an uninhibited tumor growth model, (2) model the results of external factors like immune response or treatment, and (3) model tumor and other immune resistance to treatments.
Models measure \textit{tumor burden}, denoted $T$, (see\ \ref{appendix: defs} for definitions), as a function of time $t$.
Commonly used models include linear, exponential, and logistic models (see\ \ref{appendix: models} for equations).
However, these are not accurate to the growth of most observed cancers \ \cite{Steb23}, which involve a slow exponential rate of increase and a slow convergence to a maximal tumor burden.
The purpose of these general growth models is to develop more personal cancer treatment to individuals.\ \cite{YinMoes}

Most cancers are treated through a combination of surgery, chemotherapy, and radiation therapy.
Surgery removes as much as the cancer as possible, but is unsuited for modeling because it creates discontinuities.
Therefore, although surgery is the primary treatment for breast cancer, we will focus on modeling chemotherapy (see\ \ref{appendix: defs}) since it is the most common treatment supplement to surgery.

Most models for tumor growth in response to chemotherapy show the effect of a single dose of chemotherapy on cells at specific cell-cycles and seek to model the resistance of a tumor to the given drug.
A full chemotherapy treatment cycle includes a sum of multiple doses that decay over time, and may include sudden changes in the tumor growth as modeled by Nave\ \eqref{eq: NavePersonalChemo}.
Ophir Nave described an interaction between the chemotherapy drug and both the immune system and cancer itself \ \cite{NAVE2022e09288}.
An exponential decay of the drug best models the rapid effect that chemotherapy has on the tumor burden (e.g.\ \eqref{eq:PanettaExpDec}).
Unfortunately, these models often ignore the effect of cancer treatment on other parts of the body, which we expect to have a similar response to cancer cells.
Thus, we attempt to find an expression for chemotherapy that affects both tumors and the immune system.

The immune system is the body's natural defense against foreign bacterias or other abnormal cells (including cancer cells), which includes many different cell types, of which we will focus on two.
Most models, as in\ \ref{eq: dePillisTumorImmuno} or\ \ref{eq: AlharbiTumorImmuno}, look at the response of the entire immune system to cancer, rather than the contributions of each cell type.
While working generally, it suffices to show the interaction as whole, but as mentioned by de Pillis et. al, ``in some applications, it is not sufficient to represent the immune response with a single homogeneous population of effector killer cells\ \cite{dePillis2014461}."
Thus, a good model for a certain type of cancer, should show interactions between the cells affecting that cancer.

The immune system has many types of cells, but NK cells and CD8$^+$ cells alone kill breast cancer cells\ \cite{Amens21}.
Hence modeling for tumor-immunological responses focuses on behavior of \textit{natural killer} cells (NK), which make up an innate immune system, and \textit{cytotoxic T}-cells (CD8$^+$), which form an adaptive immune system, (see\ \ref{appendix: defs}).
CD8$^+$ cells are trained to kill cancer by the NK cells, which forms a relation between NK, CD8$^+$, and cancer cells.

Immune system and tumor interactions can be modeled by \textit{Michaelis-Menten kinetics} (see\ \ref{appendix: defs}), which models interactions like immune system response to cancer cells, tumor cells affecting the immune system, or infection of normal cells by tumor cells\  \cite{math8081285}.
Any model looking into the interaction of a tumor with the immune system should consider similar interactions to these.


Given the difficulty of modeling tumor growth in general, the intricacies of chemotherapy, and the complexity of immune-tumor interactions, we can understand why creating a completely general model of any cancer is arduous.
Thus, we narrow our focus and attempt a simpler model that follows the previous guidelines.
Since the National Cancer Institute records it as having the most number of cases as of 2024, we decided to focus on breast cancer.
Our tumor growth modeling consists of a Gompertz growth model with a specific neoadjuvent chemotherapy treatment, and an immune response given by NK and CD8$^+$ cells.


%% Second Section 
\section{Modeling}

%% Finite Difference methods, feel free to edit or place where you would like --Rebecca
%% In solving the original Gompertz model, finite difference methods were used because of their simplicity yet relative accuracy. The local truncation error for Forward Euler can be found as follows $\tau_i = |T^\prime(t_i) - \frac{T(t_i + h) - T(t_i)}{h}|$  where $T(t_i)$ is the actual solution at time $i$. Using Taylor Series, we can expand this to $\Bigl |T^\prime(t_i) - \frac{T(t_i) + hT^\prime(t_i) + h^2T^\dprime(t_i) + h^3T^{\prime\prime\prime}(t_i) + O(h^4) - T(t_i)}{h}\Bigr |$ which simplifies to $|-hT^\dprime(t_i) - h^2T^{\prime \prime \prime}(t_i) + O(h^3)| = O(h)$. So although quick to code and compute, there is a relatively large error term. 


%% End Finite Difference methods
The Gompertz model is a logistic model that was created to describe the growth of human mortality in 1825 by Benjamin Gompertz.
In particular, the ODE is given by
\begin{equation}
	\frac{\diff T}{\diff t} = k_g T \ln \biggl(\frac{T_{\max}}{T} \biggr) \label{eq: gompertz},
\end{equation}
where $k_g$ is a growth constant of the tumor, $T$ is the total number of cancer cells, and $t$ is days.
The solution to the ODE is of sigmoidal nature.
Like the logistic growth model \eqref{eq: logistic}, the Gompertz model starts off with a quasi-exponential growth at the beginning that is short lived.
However, unlike the logistic model, the Gompertz model slowly converges to the carrying capacity of what a tumor can have with available nutrients.
Figure~\ref{fig:odeModels} shows how the Gompertz model differs from that of other models used for tumor growth.

\begin{wrapfigure}[18]{R}{9cm}
  \centering
    \vspace*{-42mm}
    	\begin{varwidth}{\linewidth}
		 \includegraphics[scale=0.5]{./images/ode_models.pdf}
		 \captionsetup{justification=centering, width=5cm}
		 \caption{The uninhibited growth of various ODE models}
		 \label{fig:odeModels}
	\end{varwidth}
	\vspace*{-40mm}
\end{wrapfigure}

%%\begin{figure}[htb]
%%\begin{center} %Put your images in a figure like this
%%\includegraphics[width=\textwidth]{./images/logistic.pdf} 
%%\vspace*{-50mm}
%%\end{center}
%%\caption{Tumor Burden for various growth functions}
%%\label{fig:growth} % for automatic cross referencing
%%\end{figure}
Notice how the Gompertz model slows its growth first and more significantly than a logistic model while still converging towards the carrying capacity.
The logistic growth, on the other hand, is much more aggressive after its inflection point.
For the Gompertz model, take the derivative of \eqref{eq: gompertz} and set it equal to 0, this gives us that  the inflection point of the Gompertz model is at $\frac{T_{\max}}{e}$.
This is the point when 36.8\% of the carrying capacity has been reached compared to the inflection point of the logistic model that occurs at half the carrying capacity.
As stated earlier, we wanted a model that exhibits an exponential growth but also has a slow convergence to the carrying capacity.
This is why the Gompertz is a better modeling choice than other given models\ \cite{LairdGompertz}. 

For chemotherapy effects, seeing that models are derived as exponential decays of the drug-dose and are dependent on the type of drug administered as well as the percentage of cancer killed at a specific cell-state, we opted to work with the model proposed by Bethge et al (which is similar to the one given by de Pillis and Radunskaya\ \eqref{eq: Pillis}). The chemotherapy differential expression is
\begin{equation}
	f \mu c(t) T, c(t) = e^{-\gamma t} \label{eq: chemo},
\end{equation}
where $\mu$ represents the drug sensitivity of cells (thereby implying drug effectiveness), $c(t)$ is the concentration of the given drug with a rate modeled by a decay constant $\gamma=\frac{\ln{2}}{t_{1/2}}$ after the half-life of the drug, and $f$ is is the proportion of cells that are in specific cell-cycle such that chemotherapy affects those cells specifically.
If the given drug affects all cells equally irrespective of cell cycle, then $f$ is equal to 1.

The chemotherapy differential expression specifically models the rate of change, with respect respect to time, of the death or removal of tumor cells by the given drug.
At $t=0$, we would expect a high number of cells to be killed off, and as time continues, we would expect to see that the effectiveness of the drug decaying off (hence the decay) and thereby the proportion of tumor cells killed also decreases.
Moreover, depending on how good or strong the drug is, we would expect to have a different rate of change which is the purpose of the half life in $\gamma$ and the $f$ constant.
Adding \eqref{eq: chemo} to our growth in \eqref{eq: gompertz} gives 
\begin{equation}
	\frac{\diff T}{\diff t} = k_g T \ln \biggl(\frac{T_{\max}}{T} \biggr) - f \mu c(t) =k_g T \ln \biggl(\frac{T_{\max}}{T} \biggr) - f \mu e^{-\gamma t} \label{eq: gompertz-chemo}.
\end{equation}
For our purposes, we chose to go with a chemotherapy treatment plan of one dose every two weeks for a total of 14 doses. 

Having now considered chemotherapy, we opted to also include the influence of the immune system on the tumor burden of a patient. The basis of our immune system modeling comes from an already published model which focuses on the interaction between cancer cells, NK cells, and CD8$^+$ T cells \cite{Immune}. While there are many more cells that influence tumor burden, these are the two cells that act directly on breast cancer cells \cite{Amens21}. In the following differential equations, T is the tumor cell population, N is the NK cell population, and L is the CD8$^+$ T cells population. The changes of these populations are modeled as

\begin{equation} 
	\frac{\diff T}{\diff t} = aT(1-bT) - N_{KR}NT - D \label{eq:dT}
\end{equation}
\begin{equation} \label{eq:dN}
	\frac{\diff N}{\diff t} = \sigma - N_dN +\frac{gT^2}{h + T^2}N - pNT
\end{equation}
\begin{equation} \label{eq:dL}
	\frac{\diff L}{\diff t} = - mL +\frac{jD^2}{L_R + D^2}L - qLT + rNT
\end{equation}
\begin{equation} \label{eq: D}
	D = d\frac{(L/T)^\lambda}{s + (L/T)^\lambda}T
\end{equation}

Due to the complexity and number of constants, we define and quantify them in \ Appendix \ref{appendix: models}, but we will break down the contribution of each component. 
In equation \eqref{eq:dT}, the tumor cell population increases by a single component, $aT(1-b)$, which is the logistic growth  of the tumor. The remaining two terms $N_{KR}NT$ and $D$ signify number of  tumor cells killed by NK cells and CD8$^+$ cells, respectively.
In equation \eqref{eq:dN}, the NK cell population increases due to two components. One is the constant inflow of cells independent from tumor influence, denoted $\sigma$. And the other is the increase of NK cells due to the tumor's presence. This term takes the form of a Michaelis-Menten kinetics equation, and is denoted $\frac{gT^2}{h + T^2}N$. The remaining two terms of this equation are decreases in NK cell population. The first being natural NK cell death, denoted $N_dN$, and the last one being he number of NK cells deactivated by tumor cells, denoted $pNT$.
For equation \eqref{eq:dL} we again have two components which are loses in CD8$^+$ T cells population, the terms $mL$ and $qLT$. The first of which is the number of naturally dying cells, and the second is the number deactivated by tumor cells. We also find two components contributing to an increase in CD8$^+$ T cells, the first of which being $\frac{jD^2}{k + D^2}L$, the number of CD8$^+$ cells recruited to fight cancer, again in the form of a Michaelis-Menten kinetics equation. Lastly, the term $rNT$ models the phenomenon where CD8$^+$ cells are produced after an NK cell is killed by a tumor cell. 


The combination of the two models requires some thought.
The Gompertz-Chemo integrated model for tumor burden can be added in place of the logistic growth factor in the immune system model, but the effects of the chemotherapy on the immune cells cannot be ignored.
Although sources were inconclusive on whether the chemotherapy directly killed the NK and CD8$^+$ cells, it is known that all cells are negatively affected by the therapy, mainly being inhibited in their ability to kill cancerous cells\ \cite{RebeCytoChemonImmune}.
We make the assumption that the number of NK and CD8$^+$ cells affected by the chemotherapy is proportional to\ \eqref{eq: chemo} and the number of NK and CD8$^+$ cells. 
As such, we introduced a term in the differential equations for both CD8$^+$ cells and NK cells that seeks to replicate the natural inhibition CD8$^+$ and NK cells experience.
Specifically, the term kills\footnote{We thought of this with an analogy 20 workers where only 5 are actively working, so that 15 workers are ``dead" at work. While this is a very naïve assumption, it does help exhibit inhibition.} of in proportion to their interactions with the chemotherapy drug.
In their Chemotherapy for Breast Cancer page, the American Cancer Society states that typically chemotherapy will be around 3-6 months which gives about 8-12 treatment cycles (see\ \ref{appendix: defs}).
Hence to properly show inhibition, we altered the parameter at the end of \eqref{eq: NK} and \eqref{eq: CD8} until simulations prediction at least 5 rounds of chemotherapy before \textit{remittance} (see\ \ref{appendix: defs}).
This results in the following system of equations.

\begin{equation} \label{eq:unifiedmodel}
\frac{\diff T(t)}{\diff t} = k_g T(t) \ln \biggl(\frac{T_{\max}}{T(t)} \biggr) - f \mu e^{-\gamma t} - n_{kr}N(t)T(t) - D 
\end{equation}
\begin{equation} \label{eq: NK}
\frac{\diff N(t)}{\diff t} = \sigma - n_dN(t) +\frac{g(T(t))^2}{h + T^2}N - pN(t)T(t) - N(t)f \mu e^{-\gamma t}
\end{equation}
\begin{equation} \label{eq: CD8}
\frac{\diff L(t)}{\diff t} = - mL(t) +\frac{jD^2}{l_r + D^2}L(t) - qL(t)T(t) + rN(t)T(t) - Lf \mu e^{-\gamma t}
\end{equation}
\begin{equation} \label{eq:Dconst}
D = d\frac{\Bigl(\frac{L(t)}{T(t)}\Bigr)^\lambda}{s + \Bigl(\frac{L(t)}{T(t)}\Bigr)^\lambda}T(t)
\end{equation}
The constants and form of the equation should be familiar, as they are taken directly from the above equations. 


To begin modeling, we called $t_o=t=0$ to be the time when a tumor should be of sufficient size that a person, medical practice, or immune system should be able to identify or detect the cancer.
While it is unknown why cancer can go undetected completely from the immune system or medicine, there are estimates for a tumor burden that should be detectable. 
Specifically, Caley and Jones stated that when cancer is detectable there are about $10^8$ to $10^9$ tumor cells in the body\ \cite{CALEY2012186}. 
Since it was part of our interest to examine whether our model would show a tumor growing to its $T_{\max}$ without much immunological intervention, we begun with an initial tumor burden of $T_o = 9\times10^8$ (number of cells) as shown in  Figure~\ref{fig:FullODE}.

For the chemotherapy, we opted to start the treatment at $45$ days from $t_o$ to account for the time a patient has between waiting for biopsy results, further lab tests, and the actual start of the chemotherapy.

In the paper of Sopik and Narod, they estimated the probability of dying from breast cancer as a function of the size of the tumor and found that, while some primary tumor sizes can be as big as 150$mm$ in diameter, the mortality probability plateaus starting with tumors with a diameter of  91$mm$\ \cite{SopikTumorSize}.
Specifically, the mortality probability of tumors with a  diameter of 150$mm$ is about $64.1\%$ whereas tumors with a diameter of size ranging from $91-100 mm$ exhibited a $60\%$ mortality probability.
Thus, there is not much of an increase in chance of death with a bigger tumor size.
Hence, our $T_{\max}$ became the number of cells found in a tumor of $mm$ in diameter.
Larsen et al found that, on average and in a highly dense colorectal carcinoma, there is about 35$mm^2$  of tumor tissue\ \cite{LarsenStineNkCellsRatio}.

%% Third Section
\section{Results}


%Here we have a forward Euler numerical method approximating the tumor burden with and without a single session of chemotherapy. 
%We see that without chemotherapy the cancer grows exponentially. 
%On the other hand, after a session of chemo, 
%we see a sharp decline followed by a continuation of exponential growth

%%\begin{figure}[h!]
%%\begin{center} %Put your images in a figure like this
%%\includegraphics[width=\textwidth]{./images/chemo.pdf} 
%%\vspace*{-50mm}
%%\end{center}
%%\caption{Tumor burden with vs without chemo session}
%%\label{fig:chemo} % for automatic cross referencing
%%\end{figure}

\begin{figure}[h!]
\begin{center} %Put your images in a figure like this
\includegraphics[scale=0.6]{./images/growth_8T_3W_C45.pdf} % Better to make them pdfs than png or gif or jpeg
\end{center}
\caption{The growth of breast cancer  as modeled by \eqref{eq:unifiedmodel} for 8 cycles with each one occurring every 3 weeks for $T_o=9\times 10^8, T_{\max}=9\times10^9$.}
\label{fig:FullODE} % for automatic cross referencing
\end{figure}

As mentioned earlier, starting at $(t_o, T_o)$, we can see in Figure~\ref{fig:FullODE} that our model exhibits growth for an amount that should, in theory, not yet be detectable.
Thus, initially our model does well in describing the dynamics that tumor burden should exhibit on the premise the immune system has not fully detected.
That is, the characteristics are that the tumor remains undetectable for the most part with perhaps some interactions with the immune system but not to the degree the immune system recruits heavily to attack cancer.

\begin{figure}[h!]
\centering
	\begin{subfigure}{.5\textwidth}
  		\centering
 		 \includegraphics[scale=0.43]{./images/NK_growth_8T_3W_C45_semiy.pdf}
 		 \caption{NK cells.}
 		 \label{fig:NKGrowth}
	\end{subfigure}%
	\begin{subfigure}{.5\textwidth}
  		\centering
  		\includegraphics[scale=0.43]{./images/CD8_growth_8T_3W_C45_semiy.pdf}
  		\caption{CD8$^+$ cells.}
 		 \label{fig:CD8Growth}
	\end{subfigure}
	\caption{The growth of immune cells in relation to the tumor of Fig~\ref{fig:FullODE}.}
\end{figure}

Another important aspect of our model as displayed in Figure~\ref{fig:FullODE} is the successful remission of cancer.
In Figure~\ref{fig:FullODE_semiy} (see in Appendix \ref{appendix:graphs}), we can more fully appreciate the death of cancer to tumor burden levels that are near zero.
This seems 

The fully integrated cancer model with both chemotherapy and immune cells predicts unreasonable results for chemotherapy without additional adjustments. For reasonable initial tumor sizes, one or two rounds of chemotherapy almost immediately kill the tumor. In practice, patients undergo 5-8 rounds of chemotherapy on average before showing signs of remission. Such problematic behavior is likely a result of the immune model, which as presented in a previous paper does not generalize well past 30 days. Thus, we tweaked our initial full model to produce qualitatively sound results.
 


To solve the equation given by\ \eqref{eq:unifiedmodel}, we used \begin{verbatim} scipy.integrate.solve_ivp \end{verbatim}
that use the explicit Runge-Kutta method of order 5(4) which assumes an accuracy of the fourth-order but takes steps using a fifth-order accurate formula.

Moreover, we chose to go with a chemotherapy treatment plan of one dose every four weeks (i.e.\ 28 days) for a total of 14 doses.
Specifically, those doses started about 60 days after the initial discovery of tumor.
The results are shown in\ Figure~\ref{fig:FullODE}.
These results seem to be on par with the estimates given by Mayo Clinic where for advanced breast cancer, treatment is beyond that of six months.

Here we plot the differential equations of the immune system. 
We get our initial values from a plot in the paper where we got the differential equations. 
In the paper published, the populations only go to about 35 days. 
Below we've created similar plots, but as we increase the time beyond 35 days, 
we start to see some problems, which we will address in the analysis/conclusions section.

\begin{figure}[h!]
\begin{center} %Put your images in a figure like this
\includegraphics[scale=.6]{./images/forward_immune.pdf} % Better to make them pdfs than png or gif or jpeg
\end{center}
\caption{The Forward Euler method of each cell population generated for 2 sets of initial conditions}
\label{fig:forward} % for automatic cross referencing
\end{figure}





%%Fourth Section
\section{Analysis/Conclusions}

Weakness: cannot model properly for bigger Tmax anything from 50mm on

One of the hard aspects was that for certain values our modeling would show quite the decline.
ADD TODO graph in supplemental information.
These would 

Our initial numerical method approximating the tumor burden with and without chemo showed the sharp decline that a chemo session causes in the tumor cell population. 
On a long enough time scale we would see a plateau in the chemo population growth as it reaches its carrying capacity. 
The challenge with this dynamic is that this only models a single chemo session, 
while we would expect many round of several sessions each in the treatment of cancer.

In modeling the immune system alone, we had some success, 
but the dynamics we were able to capture don't work well on a long enough time scale. 
Specifically, after about 35 days we start to see negative populations and discontinuous changes 
in population. While we may be able to gain some insight on the interaction between cancer cells, 
Natural Killer cells, and CD8$^+$ cells, we have an unreasonably short time scale that our model works on. 

Our full model produces reasonable results, but the immune system component of the model remains a weak point. Qualitatively, our model performs as expected when we include chemotherapy. However, the immune system component remains too strong when left on its own. Only by including the effects of chemotherapy on the immune system did we keep the strength of the CD8$^+$ cells in check. If our model is to be useful, it needs to accurately predict tumor burden in the absence of chemotherapy.

If our model were accurate in predicting tumor burden alone, it might be useful to prescribe chemotherapy treatments to breast cancer patients. Given initial measurements of tumor burden and immune cells, doctors could verify whether a patient is in remission. 

Unfortunately, on the other end of the table, results of treatment are not always positive. Our model might indicate whether chemotherapy is a viable cure to breast cancer of a certain size given immune system conditions. However, currently, our model does not predict negative outcomes well. Given more time, we might figure out constants for which certain initial conditions produce an overtake by cancer cell growth and thus an equilibrium solution representing a bad outcome.

\newpage

\appendix
\section{Definitions}
\label{appendix: defs}
The following definitions are derived from the National Cancer Institute, unless otherwise stated
\begin{itemize}
	\item Adaptive Immune System: the part of the immune system that specifically targets the germs or foreign substances that are causing an infection. In order to do this, this system needs to first recognize the substance as such. Therefore, this system is slower and needs training. CB8$^+$ cells are part of this system.
	\item Cancer:  a term for diseases in which abnormal cells divide without control and can invade nearby tissues
	\item Chemotherapy: a cancer treatment where drugs are used to kill cancer cells or stop them from dividing
		\begin{itemize}
			\item Neoadjuvent Chemotherapy: chemotherapy administered before the primary treatment of the tumor is performed. Typically, surgery is the primary treatment. Its main goal is to shrink the tumor so that it is easier to remove.
			\item Adjuvent Chemotherapy: Chemotherapy administered after primary tumor treatment is administered. Its intent is to lower the risk of the cancer returning.
		\end{itemize}
	\item Cytotoxic/CD8$^+$ T cell: is a T-lymphocyte that kills or infected cells or cells that are damaged in other ways. They are not natural killers and as such have to be trained to kill cancer. (Mayo clinic)
	\item Innate Immune System: the part of the immune system that is the first line of defense against intruders or unknown foreign cells in the body. It responds to all foreign substances in the same manner (National Library of Medicine). It can be thought of as "kill first, ask questions later." NK cells are part of this system.
	\item Log-kill Hypothesis: when growth of a cancer is exponential—increasing by a constant fraction of itself every fıxed unit of time—then in the presence of effecive anticancer drugs it also shrinks by a constant fraction \cite{LogKill}
of itself
	\item Remission: A decrease in or disappearance of signs and symptoms of cancer. In partial remission, some, but not all, signs and symptoms of cancer have disappeared. In complete remission, all signs and symptoms of cancer have disappeared, although cancer still may be in the body.
	\item Treatment cycle: the regular and repeated interval of time between each new dose of a chemotherapy drug. A cycle comprises of a rest period to allow the body to heal from the effects until the new dose is given. This information was retrieved from the American Cancer Society and\ \cite{CALEY2012186}.
	\item Tumor: an abnormal mass of tissue that forms when cells grow and divide more than they should or do not die when they should. Tumors may be \textit{benign} (not cancer) or \textit{malignant} (cancer). For this project, defined the tumor burden as the number of cancer cells in the body.
	\item Tumor burden: the size of a tumor or number cancer cells. This is the total amount of cancer found in the body.
        \item Natural Killer Cell (NK Cell): A type of immune cell that has granules (small particles) with enzymes that can kill tumor cells or cells infected with a virus. A natural killer cell is a type of white blood cel
        \item Michaelis-Menten kinetics: Equations from biochemistry used to model enzymatic reaction rates, or the rate at which an enzyme acts upon some molecule to form a complex and then act in such a way so as to produce a new product and regenerate the original enzyme.
\end{itemize}

\section{Models}
\label{appendix: models}
\begin{itemize}

	\item Tumor Growth Models:
		\begin{itemize}
	\item Linear growth: 
		\begin{equation}
			\frac{\diff T}{\diff t} = k,
			\label{eq: lin}
		\end{equation}
		where $k$ is the growth rate
	\item Exponential Growth:
		\begin{equation}
			\frac{\diff T}{\diff t} = kT \label{eq: exp}
		\end{equation}
	 or with a death rate constant of $d$, $\frac{\diff T}{\diff t} = (k-d)T$
	\item Logistic Growth: 
		\begin{equation}
			\frac{\diff T}{\diff t} = kT \biggl(1- \frac{T}{T_{\max}}\biggr)\label{eq: logistic},
		\end{equation}
		where $T_{\max}$ is the max size a tumor can be, which is equivalent to the carrying capacity.
		\end{itemize}
		
	\item Chemotherapy Models: 
		\begin{itemize}
			\item Exponential Decay: Pillis and Radunskaya modeled the mix of immunotherapy and chemotherapy on tumor growth. In particular, they modeled the drug as an exponential decay given by 
				\begin{equation}
					G_M = -\gamma M \label{eq: Pillis},
				\end{equation}
				where $M=M(t)$ is the concentration of the drug in the bloodstream at some time $t$.
				
			\item Panetta also used an exponential but considering the frequency between doses as
				\begin{equation}
					g(t) = h e^{-\gamma(t-n-\tau)} \label{eq:PanettaExpDec},
				\end{equation}
				where $g(t)$ is the effects of the chemotherapy drug, $\gamma$ is the decay of the drug, $n$ is number of doses, and $\tau$ is the period between doses.
				
			\item Personalized treatment: Ophir Nave modeled a personalizable treatment plan as 
				\begin{equation}
					\mathscr{F} = \sum_{k=0}^n q(t-mk) \mathscr{H} (t-mk)e^{\frac{t-mk}{0.5}}\label{eq: NavePersonalChemo},
				\end{equation}
				where $n$ is the duration of the treatment, $m$ is the interval between treatments, and $\mathscr{H}$ a unit step function.
		\end{itemize}
	\item Immunological Response Models:
		\begin{itemize}
			\item Pillis, Radunskaya, Wiseman:\\
                    			$\frac{dT}{dt} = aT(1-bT) - cNT - D\\$
                    			$\frac{dN}{dt} = \sigma - fN +\frac{gT^2}{h + T^2}N - pNT\\$
                    			$\frac{dL}{dt} = - mL +\frac{jD^2}{k + D^2}L - qLT + rNT\\$
                    			$D = d\frac{(L/T)^\lambda}{s + (L/T)^\lambda}T$\\
                    		Where we define each constant:
                   		 \begin{itemize}
                    			\item a = $5.14 \times 10^{-1}$ has units $\text{day}^{-1}$ is the tumor growth rate
                    			\item b = $1.02 \times 10^{-9}$ has units $\text{cell}^{-1}$ where  $\frac{1}{b}$ is the tumor carrying capacity.
                   		 	\item $N_{NR}$ = $3.23 \times 10^{-7}$ has units $\text{cell}^{-1}\text{day}^{-1}$ is the fractional cell kill(see appendix) rate of NK cells against tumors.  
                   		 	\item $sigma = 1.3 \times 10^4$ has units $\text{cells} \text{day}^{-1}$ is the constant NK cells production.
                   		 	\item $N_d = 4.12 \times 10^{-2}$ has units $\text{day}^{-1}$ is the natural death rate of NK cells.
                   		 	\item $g = 2.5 \times 10^-2$ has units $\text{day}^{-1}$ is the max NK recruitment
                   		 	\item $h = 2.02 \times 10^7$ has units $\text{cell}^2$ is the steepness coefficient of the NK recruitment curve.
                   		 	\item $p = 1.00 \times 10^{-7}$ has units $\text{cell}^{-1}\text{day}^{-1}$ is the rate at which tumors incapacitate NK cells
                   		 	\item $m = 2.00 \times 10^{-2}$ has units $\text{day}^{-1}$ is the natural death rate of CD8$^+$ cells.
                   		 	\item $j = 3.75 \times 10^{-2}$ has units $\text{day}^{-1}$ is the max CD8$^+$ recruitment rate, and the constant $k = 2 \times 10^7$ has units $\text{cell}^2$ is the steepness coefficient of the CD8+ recruitment curve.
                   		 	\item $L_R = 2 \times 10^7$ has units $\text{cell}^2$ is the steepness coefficient of the CD8$^+$ recruitment curve.
                   		 	\item $q = 3.42 \times 10^{-10}$ has units $\text{cell}^{-1}\text{day}^{-1}$ is the rate that tumors deactivate CD8$^+$ cells.
                   		 	\item $r =1.1 \times 10^{-7} $ has units $\text{cell}^{-1}\text{day}^{-1}$ is the rate at which those CD8$^+$ cells are produced. 
                    			\item $d = 5.80$ has units $\text{day}^{-1}$ is the saturation level of fractional tumor cell kill by CD8$^+$ T cells
                    			\item $s = 2.5 \times 10^{-1}$ has no units, and is the steepness of the curve which determines the Tumor vs. CD8$^+$ cell competition. Lastly, \item $\lambda = 1.36$ has no units. 
				\end{itemize}
			\item Alharbi \& Sham Rambely: their modeling equations looked at the interaction of tumor cells and the immune system, $I$, as a whole as well as normal cells , $N$, (non-immune, non-tumor cells). They described the relationships by (using a logistic growth for tumor $T$):
				\begin{eqnarray}
					\begin{aligned}
						\frac{\diff N}{\diff t} &= rN(1-\beta_1 N) - \eta NI - \gamma NT \\
						\frac{\diff T}{\diff t} &= \alpha_1T(1-\alpha_2T) + \beta_2 NT - \alpha_3 T1 \\
						\frac{\diff I}{\diff t} &= \sigma - \delta I _ \frac{\rho N I}{m+N} + \frac{\rho_1 TI}{m_1 + T} - \mu NI - \mu_1 TI \label{eq: AlharbiTumorImmuno}
					\end{aligned}
				\end{eqnarray}
			\item dePillis et. al: they modeled the primary interaction between effector cells, $E$, like CB8$^+$, and the tumor, $T$ by using logistic growth and 
				\begin{eqnarray}
					\begin{aligned}
						\frac{\diff T}{\diff t} &= a_1T(1-b_1T) - c_2ET - c_3NT - k_2(1-e^{-u}) \\
						\frac{\diff N}{\diff t} &= a_2(1-b-2N) - c_4NT - k_3 (1-e^{-u})\label{eq: dePillisTumorImmuno}
					\end{aligned}
				\end{eqnarray}
		\end{itemize}
	\item Growth-Chemo-Immune PDE System: Ansarizadeh, Singh, and Richards modeled tumor cells using a system of PDEs. Specifically, they used a logistic model for the normal cells $N$, tumor $T$, immune $I$, and the chemotherapeutic drug $U$. For them, the drug was only active for certain phases of the cell division cycle the expression $1-e^{-U}$ was used to denote the fraction of cells killed.
		\begin{eqnarray}
			\begin{aligned}
				\frac{\partial N}{\partial t} &= r_2 N (1-b_2)N - c_4TN - a_3(1-e^U)N + D_N \frac{\partial^2 N }{\partial x^2} \\
				\frac{\partial T}{\partial t} &= r_1 N (1-b_1 T) - c_2 IT - c_3TN - a_2(1-e^{-U})T + D_T \frac{\partial^2 T }{\partial x^2} \\
				\frac{\partial I}{\partial t} &= s + \frac{\rho IT}{\alpha + T} - c_1 IT - d_1 I - a_1(1-e^{-U})I +D_I \frac{\partial^2 I }{\partial x^2} \\
				\frac{\partial U}{\partial t} &= v(t) -d_2U + D_U \frac{\partial^2 U}{\partial x^2}\label{eq:GrowthChemoImmunoPDE}
			\end{aligned}
		\end{eqnarray}
\end{itemize}

\section{Supplemental Graphs}
\label{appendix:graphs}
In this section, we give more graphs that help our analysis as supplementary information to the main points and graphs given in the paper.
\begin{figure}[h!]
\begin{center} %Put your images in a figure like this
\includegraphics[scale=0.6]{./images/growth_8T_3W_C45_semiy.pdf} % Better to make them pdfs than png or gif or jpeg
\end{center}
\caption{The growth of breast cancer, in a semilog scale for $T$, as modeled by \eqref{eq:unifiedmodel}. Compare to Figure~\ref{fig:FullODE}. This graph helps us appreciate the actual death of cancer to near zero.}
\label{fig:FullODE_semiy} % for automatic cross referencing
\end{figure}
%%%%%%%%%%%%%%%%%%%%%%%%%%%%%%%%%%%%%
%% Bibliography below
%%%%%%%%%%%%%%%%%%%%%%%%%%%%%%%%%%%%%
%\FloatBarrier % Keep the figures from being put after the bibliography
\newpage
%% If using bibtex, leave this uncommented
\bibliography{refs.bib} %if using bibtex, call your bibtex file refs.bib
\bibliographystyle{alpha}
\nocite{*}
%% If not using bibtex, comment out the previous two lines and uncomment those below
%\begin{thebibliography}{99}
%\bibitem{Vandermeersch} First reference goes here
%\end{thebibliography}
\end{document}
