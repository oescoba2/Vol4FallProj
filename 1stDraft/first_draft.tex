\documentclass[11pt]{amsart}
%%% WARNING: Do NOT change the page size, fonts, or margins!  Penalties will apply.


\usepackage{graphicx}
\usepackage{amssymb, amsmath, amsthm, cite}
%\usepackage{places} %enables \FloatBarrier, which prevents figures and tables from going below it.
%\usepackage{hyperref} %makes cross references into hyperlinks. 
\usepackage{amsmath, amssymb, amsthm, amsfonts, algpseudocode, algorithm, bbm, color, fixmath, float, graphicx, hyperref, listings, mathrsfs, mathtools, subfig, times}

%Needed commands
\newcommand*{\w}{\mathbf{w}}
\newcommand*{\x}{\mathbf{x}}
\newcommand*{\y}{\mathbf{y}}
\newcommand*{\z}{\mathbf{z}}
\newcommand*{\R}{\mathbb{R}}
\newcommand*{\E}{\mathbb{E}}
\newcommand*{\0}{\mathbf{0}}
\newcommand*{\minimizer}{\mathbf{x}^*}
\newcommand*{\dprime}{{\prime\prime}}
\newcommand{\li}[1]{\lstinline[prebreak=]!#1!}
\newcommand{\pseudoli}[1]{\lstinline[style=pseudo]!#1!}
\newcommand{\trp}{^{\mathsf T}} 
\newcommand{\im}{{i\mkern1mu}}
\newcommand{\Real}{\mathchardef\Re="023C}
\newcommand{\Imag}{\mathchardef\Im="023D}
\newcommand\norm[1]{\left\lVert#1\right\rVert}
\newcommand*\diff{\mathop{}\!\mathrm{d}}
\newcommand*\Eval[3]{\left.#1\right\rvert_{#2}^{#3}}

%Operators
\DeclareMathOperator{\argmin}{argmin}
\DeclareMathOperator{\argmax}{argmax}

%Link set up
\hypersetup{
    colorlinks=true, %set true if you want colored links
    linktoc=all,     %set to all if you want both sections and subsections linked
    linkcolor=blue,  %choose some color if you want links to stand out
    pdftitle={RL Notes},
    pdfpagemode=FullScreen
}

\endinput

%%% WARNING: Do NOT change the page size, fonts, or margins!  Penalties will apply.
%%% WARNING: Do NOT change the page size, fonts, or margins!  Penalties will apply.
\title{The Effect of Chemotherapy and Immune System Response on Breast Cancer Growth}
\author{Rebecca Gee, Henry Fetzer, Nephi Suyama, Joseph Humpherys, Oscar J. Escobar}
\date{October 29 2024} % or use \today

\begin{document}
\maketitle % this actually makes the title

\begin{abstract}
Place abstract here. The abstract summarizes in one paragraph the main question and conclusions draw from your investigation.
\end{abstract}

%% First Section
\section{Background/Motivation}

Mathematical oncology is field of oncology, the study of cancer, that employs math to study cancer and its behavior.
In the words of Dr. Rockne and MD Scott, ``it [serves] as a bridge between $\ldots$ the biologist, and the practicing clinician (2019)."
Some of the most recent and important reasons for math modeling in oncology is to understand and model the characteristics and growth of cancer.
Moreover, it seeks also to understand and model the relationship between cancer and the immune system and/or its response to treatment or resistance to it.

Tumor growth modeling is a well researched area of mathematical oncology.
Its main purpose is to model tumor growth without any intervention as well as growth in response to external factors such as immunological response or treatment. 
In the absence of any intervention, several models have been made to try to show the growth of a tumor, measured by \textit{tumor burden}, denoted $T$, (see \ref{appendix: defs} for definitions), as a function of time $t$.
The models range from simple ODEs such as linear growth, logistic growth, to more complicated models employing stochastic differential equations and algebraic differential equations (which would be outside the scope of material learned in Vol 4 for analysis). 
The main hope is to be able to use these models to develop more personable treatment to individuals facing the plight of cancer\ \cite{YinMoes2019}.

The more commonly used models due to their simplicity are linear, exponential, and logistic models (see\ \ref{appendix: models} for equations).
However, these do not accurately reflect the full and overall growth of observed cancers with the exception of a few cancer scenarios.
That is, they fail to generalize to the dynamics a cancer will exhibit: primarily that it has slow exponential rate and a maximal size.
In particular, the exponential model\ \eqref{eq: exp} is characterized by an infinite growth as $t$ increases which does not reflect the fact that a tumor can have a maximum size, even when considering a death rate.
Moreover, the logistic model\ \eqref{eq: logistic} converges too fast to the max size, $T_{\max}$, a tumor can be\ \cite{Steb23}.
As such, a need for a model that can firstly exhibit a slow exponential rate and then a slow convergence to the carrying capacity is preferred to others that only exhibit one of these two characteristics.

When it comes to treatment, most cancers typically use a combination for surgery, chemotherapy, and radiation therapy for treatment.
In breast cancer, surgery is the primary treatment which is not well suited for math modeling. 
Surgery removes as much as the cancer as possible so that any modeling growth would just have a sudden vertical drop in tumor burden at the time the surgery is removed causing discontinuities in the modeling.
For breast cancer, chemotherapy is the most common treatment supplement both before and after surgery.

Most models for tumor growth in response to chemotherapy are primarily based on chemotherapy affecting cells at specific cell-cycles but mainly seeking to model the resistance of a specific tumor to the given drug or drugs.
The few generalized models are more of the nature of exponential decay of the administered drug to the patient. 
The tumor-chemotherapy models most often only seek to understand the behavior of a tumor in response to the treatment and ignore any underlying work that the immune system is already performing to fight the cancer.

On the other hand, the immune system naturally patrols the body in search of foreign bodies to kill and prevent diseases.
This patrolling involves not just for foreign bacteria but also abnormal cells such as cancer cells. 
The modeling for tumor-immunological response typically look at relationship between NK and CB8$^+$ (see\ \ref{appendix: defs})
TODO Add graph showing the different models

THIS IS A TEST TO SEE IF BRANCHING WORKS

Thus, our focus is to model the growth of a HER2 positive breast cancer in relation to immunological system response of the Nk and CB cells and under neoadjuvent (see \ref{appendix: defs}) chemotheraphy.

%\begin{figure}[htb]
%\begin{center} %Put your images in a figure like this
%\includegraphics[width=\textwidth]{Myfig.pdf} % Better to make them pdfs than png or gif or jpeg
%\end{center}
%\caption{Plots should be high resolution (pdf 300dpi), uncluttered, a reasonable size, and easily readable and understandable.  All figures should have a complete caption that helps the reader make sense of the figure even if they haven't read the paper yet. Here is an example: This figure shows the risk or mean squared error (in black) of for a generic family of regression models as a function of model complexity (the number of free parameters in the model).  The generalized aliasing decomposition shows that the risk is the sum of three parts: Model insufficiency (red), Data insufficiency (green), and Aliasing (blue).  Model insufficiency is monotonically decreasing as a function of complexity, Data insufficiency vanishes when the number of parameters is less than the number of training points (the classical regime), but increases monotonically in the modern regime (where the number of parameters is greater than the number of data points).  Aliasing generally increases up to the interpolation threshold and decreases thereafter, converging to zero almost surely as the number of parameters goes to infinity.  
%}
%\label{fig:MeanSquaredError} % for automatic cross referencing
%\end{figure}



%% Second Section 
\section{Modeling}

The \textit{Gompertz} model is a logistic model that was created to describe the growth of human mortality in 1825 by Benjamin Gompertz.
In particular, the ODE is given by
\begin{equation}
	\frac{\diff T(t)}{\diff t} = k_g T \ln \biggl(\frac{T_{\max}}{T} \biggr) \label{eq: gompertz},
\end{equation}
where $k_g$ is a growth constant of the tumor, $T$ is the total number of cancer cells, and $t$ is days.
The solution to the ODE is of sigmoidal nature.
Like the logistic growth model \eqref{eq: logistic}, the Gompertz model starts off with a quasi-exponential growth at the beginning that is short lived.
However, unlike the logistic model, the Gompertz model slowly converges to the carrying capacity of that a tumor can have with available nutrients.
That is, the Gompertz model slows its growth first and more significantly than a logistic model while still converging, slowly, towards the carrying capacity \cite{Steb23} .
Getting the derivative of \eqref{eq: gompertz} and setting it equal to 0, gives us that  the inflection point of the Gompertz model is at $\frac{T_{\max}}{e}$.
This is the point when 36.8\% of the carrying capacity has been reached compared to the inflection point of the logistic model that occurs at half the carrying capacity.
Given these characteristics, the Gompertz model is a popular and good choice for modeling tumor growth.

For chemotherapy effects, seeing that models are derived as exponential decays of the drug-dose and are dependent on the type of the type of drug administered as well as the percentage of cancer killed at a specific cell-state, we opted to work with the model proposed by  Bethge et al (which is similar to the one given by de Pillis and Radunskaya \eqref{eq: Pillis}). The chemotherapy differential expression is 
\begin{equation}
	f \mu c(t) T, c(t) = e^{-\gamma t} \label{eq: chemo},
\end{equation}
where $\mu$ represents the drug sensitivity of cells (thereby implying drug effectiveness), $c(t)$ is the concentration of the given drug with a rate modeled by a decay constant $\gamma=\frac{\ln{2}}{t_{1/2}}$ after the half-life of the drug, and $f$ is is the proportion of cells that are in specific cell-cycle such that chemotherapy affects those cells specifically.
If the given drug affects all cells equally irrespective of cell cycle, then $f$ is equal to 1.

The chemotherapy differential expression specifically models the rate of change in respect to time of the death or removal of tumor cells by the given drug.
At $t=0$, we would expect a high number of cells to be killed off, and as time continues, we would expect to see that the effectiveness of the drug levels off (hence the decay).
Moreover, depending on how good or strong the drug is, we would expect to have a different rate of change which is the purpose of the half life in $\gamma$ and the $f$ constant.
Adding \eqref{eq: chemo} to our growth in \eqref{eq: gompertz} gives 
\begin{equation}
	\frac{\diff T(t)}{\diff t} = k_g T \ln \biggl(\frac{T_{\max}}{T} \biggr) - f \mu c(t) =k_g T \ln \biggl(\frac{T_{\max}}{T} \biggr) - f \mu e^{-\gamma t} \label{eq: gompertz-chemo}.
\end{equation}
For our purposes, we chose to go with a chemotherapy treatment plan of one dose every two weeks for a total of 14 doses. 



 The primary aspect of the project is the modeling of the chosen phenomenon. If your group's repeated attempts resulted in abject failure, or your group succeeded, detail them in this section. Be sure to account for the various attempted models and why they were not appropriate. Include numerical simulations for each attempted model.  Reference figures and plots, like Figure~\ref{fig:MeanSquaredError}.


%% Third Section
\section{Results}

Clearly and succinctly state and describe the conclusions that you can draw from the model you have achieved (or the many failed attempts). Does your model(s) perform well quantitatively or qualitatively?

%%Fourth Section
\section{Analysis/Conclusions}

Discuss the appropriateness of the techniques/methods you employed in modeling. Did your group appropriately model the chosen phenomenon? If not, what different steps could you have taken if you had more time? What did you learn about the techniques/method that were used in the group project? If your model was successful, what additional insight/conclusions could you obtain from it? For instance, if you had a successfully modified SIR model, how might it affect different government policy? If you had a successful model for the spread of inaccurate information on social media, how might it be implemented to help reduce the spread of inaccurate information?



This part should all be done before you get to \emph{page 11}.  The bibliography can spill on to page 11, but we won't read text that goes past page 10.

\appendix
\section{Definitions}
\label{appendix: defs}
The following definitions are derived from the National Cancer Institute, unless otherwise stated
\begin{itemize}
	\item Chemotherapy: a cancer treatment where drugs are used to kill cancer cells or stop them from dividing
		\begin{itemize}
			\item Neoadjuvent Chemotherapy: chemotherapy administered before the primary treatment of the tumor is performed. Typically, surgery is the primary treatment. Its main goal is to shrink the tumor so that it is easier to remove.
			\item Adjuvent Chemotherapy: Chemotherapy administered after primary tumor treatment is administered. Its intent is to lower the risk of the cancer returning.
		\end{itemize}
	\item Cancer:  a term for diseases in which abnormal cells divide without control and can invade nearby tissues
	\item Cytotoxic/CB8$^+$ T-cell: is a T-lymphocyte that kills or infected cells or cells that are damaged in other ways. They are not natural killers and as such have to be trained to kill cancer.
	\item Log-kill Hypothesis: when growth of a cancer is exponential—increasing by a constant fraction of itself every fıxed unit of time—then in the presence of effecive anticancer drugs it also shrinks by a constant fraction \cite{LogKill}
of itself
	\item Tumor: an abnormal mass of tissue that forms when cells grow and divide more than they should or do not die when they should. Tumors may be \textit{benign} (not cancer) or \textit{malignant} (cancer). For this project, defined the tumor burden as the number of cancer cells in the body.
	\item Tumor burden: the size of a tumor or number cancer cells. This is the total amount of cancer found in the body.
        \item Natural killer cell (NK cell): is a white blood cell that destroy infected cells and cancer cells in the body. 
\end{itemize}

\section{Models}
\label{appendix: models}
\begin{itemize}

	\item Tumor Growth Models
		\begin{itemize}
	\item Linear growth: 
		\begin{equation}
			\frac{\diff T}{\diff t} = k,
			\label{eq: lin}
		\end{equation} 
		where $k$ is the growth rate
	\item Exponential Growth:
		\begin{equation}
			\frac{\diff T}{\diff t} = kT \label{eq: exp}
		\end{equation} 
	 or with a death rate constant of $d$, $\frac{\diff T}{\diff t} = (k-d)T$
	\item Logistic Growth: 
		\begin{equation} 
			\frac{\diff T}{\diff t} = kT \biggl(1- \frac{T}{T_{\max}}\biggr)\label{eq: logistic},
		\end{equation}
		where $T_{\max}$ is the max size a tumor can be, which is equivalent to the carrying capacity.
		\end{itemize}
		
	\item Chemotherapy: Pillis and Radunskaya modeled the mix of immunotherapy and chemotherapy on tumor growth. In particular, they modeled the drug as an exponential decay given by 
		\begin{equation}
			G_M = -\gamma M \label{eq: Pillis},
		\end{equation}
		where $M=M(t)$ is the concentration of the drug in the bloodstream at some time $t$.
\end{itemize}


%%%%%%%%%%%%%%%%%%%%%%%%%%%%%%%%%%%%%
%% Bibliography below
%%%%%%%%%%%%%%%%%%%%%%%%%%%%%%%%%%%%%
%\FloatBarrier % Keep the figures from being put after the bibliography
\newpage
%% If using bibtex, leave this uncommented
\bibliography{refs.bib} %if using bibtex, call your bibtex file refs.bib
\bibliographystyle{alpha}

%% If not using bibtex, comment out the previous two lines and uncomment those below
%\begin{thebibliography}{99}
%\bibitem{Vandermeersch} First reference goes here
%\end{thebibliography}

\end{document}